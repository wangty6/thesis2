%%
% 引言或背景
% 引言是论文正文的开端,应包括毕业论文选题的背景、目的和意义;对国内外研究现状和相关领域中已有的研究成果的简要评述;介绍本项研究工作研究设想、研究方法或实验设计、理论依据或实验基础;涉及范围和预期结果等。要求言简意赅,注意不要与摘要雷同或成为摘要的注解。
% modifier: 黄俊杰(huangjj27, 349373001dc@gmail.com)
% update date: 2017-04-15
%%

\chapter{引言}
%定义,过去的研究和现在的研究,意义,与图像分割的不同,going deeper
\label{cha:introduction}
\section{选题背景与意义}
\label{sec:background}
% What is the problem
% why is it interesting and important
% Why is it hards, why do naive approaches fails
% why hasn't it been solved before
% what are the key components of my approach and results, also include any specific limitations,do not repeat the abstract
%contribution
引言是论文正文的开端,应包括毕业论文选题的背景、目的和意义;对国内外研究现状和相关领域中已有的研究成果的简要评述;介绍本项研究工作研究设想、研究方法或实验设计、理论依据或实验基础;涉及范围和预期结果等。要求言简意赅,注意不要与摘要雷同或成为摘要的注解。

现代应用数学在向着“纯粹”和”应用“两大方向发展。即在最为抽象的领域中不断取得到优美定理的同时,又不断提供给应用科学和工程领域以解决问题的强有力的现代工具。将最新的抽象模型与计算成果应用于实际问题而得到具有指导意义的结果,是现代应用数学的一大特征。十分明显的,应用数学已经在生命科学,工程学以及计算机科学方面发挥着越来越重要的作用,抽象代数的引入为这些交叉学科问题不断提供着创新的解法和令人耳目一新的模型。

其中,Lotka-Volterra方程便是一个极为经典的例子。Lotka-Volterra方程的本质是n维空间上的微分方程与动力学建模。基于Lotka-Volterra方程的推导,定理,模型甚至一些猜想从第一次世界大战后直到今天一直都在蓬勃发展。

Lotka-Volterra方程其实并不是Lotka和Volterra合作提出的。而是最初由Alfred J. Lotka在1925年提出的,最初是描述植物物种和食草动物物种的种群关系。同样的方程组在1926年由数学物理学家Vito Volterra发表。Volterra的成果很大一部分是与海洋生物学家Umberto D'Ancona的共同实验得出的。D'Ancona研究了Adriatic海区的渔获量,并注意到在第一次世界大战期间捕获的掠食性鱼的比例有所增加(1914-1918)。这使他感到困惑,因为在战争期间,捕鱼的工作已经大大减少了。Volterra独立于Lotka开发了种群模型。有趣的是D'Ancona当时正在追求Volterra的女儿,公式发表之后D'Ancona迎娶了Volterra的女儿。

在1998年国际数学家大会召开时,Lotka-Volterra模型又迎来了一个里程碑式的进展。奥地利数学家J.Hofbauer和Karl.Sigmund将Lotka-Volterra微分方程与动力系统理论结合,提出了平均Liapunov函数方法。K.Sigmund以此为主题,在国际数学家大会上做了一小时全会报告。此后在1998到2010年间,基于Liapunov函数的Lotka-Volterra模型在应用数学领域得到了极大的发展。应用数学家们致力于研究基于Lotka-Volterra模型的常微系统的整体稳定性与持续生存性等问题,以及n维模型解析解的分析和讨论。比如最近提出的实根分离算法应用于三种群竞争系统极限环的算法化构造,得到了三个极限环的存在性证明。

可以说,最近十年里,以Lotka-Volterra方程为核心的应用数学分支发展突飞猛进,在生态学,博弈学,经济学,乃至社会科学都有很多应用。这种交叉学科的应用也很好解释:利用动力学的手段建模是非常合理的,真实生活行为的每种形式都是经过反复实验而成型的;这种逐步的适应性和通过个体的学习甚至自然选择都可以通过微分方程与动力学模拟。用数学语言表达即是:用未知量表征系统内每个对象,通过在连续时间上建立相互关系方程组,理论上就可以计算出任何时间的全部对象的状态。

Lotka-Volterra方程组及其衍生数学模型描述了系统中竞争,合作等不同模式的相互关系。有趣的是,当方程组的维数n=3时,已经没有解析解,我们只能通过分析解的性态

\section{国内外研究现状和相关工作}
\label{sec:related_work}
对国内外研究现状和相关领域中已有的研究成果的简要评述;
\section{本文的论文结构与章节安排}

\label{sec:arrangement}
本文共分为五章,各章节内容安排如下:

第一章引言。

第二章动力学模型,稳定性分析与分岔理论。

第三章两种群的捕食方程,竞争方程,解的性态。

第四章n种群的捕食方程,竞争方程,解的性态。

第五章应用一:基因调控网络

第六章应用二:竞争型神经网络与医学图像处理

第七章是本文的最后一章,总结与展望。是对本文内容的整体性总结以及对未来工作的展望。

